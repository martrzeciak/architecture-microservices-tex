\cleardoublepage

\chapter{Orkiestracja kontenerów i rozwiązania chmurowe}
\label{cha:OrkiestracjaKontenerow}

\section{Kubernetes w ekosystemie .NET}
\label{sec:Kubernetes}

\subsection{Azure Kubernetes Service (AKS)}
\label{subsec:AzureAKS}

AKS jest wiodącą platformą dla deploynetu mikroserwisów .NET w chmurze Azure. Ten podrozdział uzasadnia się przez praktyczne potrzeby wdrażania i zarządzania mikroserwisami w środowisku produkcyjnym. AKS oferuje zarządzaną płaszczyznę kontrolną Kubernetes oraz integrację z innymi usługami Azure.

\subsection{Wdrażanie z Helm Charts}
\label{subsec:HelmCharts}

Helm Charts stanowią standard dla wdrażania aplikacji w Kubernetes. Ten podrozdział uzasadnia się przez konieczność pokazania, jak automatyzować deployment mikroserwisów oraz zarządzać ich konfiguracją w różnych środowiskach. Helm umożliwia wersjonowanie i przywracanie poprzednich wersji wdrożeń, co jest kluczowe dla CI/CD.

\section{Service Mesh}
\label{sec:ServiceMesh}

\subsection{Linkerd i Istio}
\label{subsec:LinkerdIstio}

Service Mesh rozwiązuje problemy komunikacji, bezpieczeństwa i observability w architekturze mikroserwisowej. Porównanie Linkerd i Istio uzasadnia się przez znaczące różnice w wydajności - Linkerd dodaje 40-400\% mniej latencji niż Istio. Ten podrozdział jest istotny dla zrozumienia, jak Service Mesh wpływa na wydajność komunikacji między mikroserwisami.

\subsection{Distributed Tracing}
\label{subsec:DistributedTracing}

Distributed Tracing jest niezbędny dla debugowania i monitorowania systemów mikroserwisowych.

\section{CI/CD Pipeline dla mikroserwisów}
\label{sec:CICD}

CI/CD dla mikroserwisów wprowadza unikalne wyzwania związane z niezależnym deploynmentem serwisów. Azure DevOps oferuje dedykowane narzędzia dla tego celu.
