\cleardoublepage %wymuszenie umieszczenia na nieparzystej stronie (dotyczy to tylko wstępu, pierwszego rozdziału, podsumowania i~bibliografii)
\chapter*{Wstęp}
\addcontentsline{toc}{chapter}{Wstęp}

%--------------------------------------
% początek: wskazówka
%--------------------------------------
%„Wstęp powinien przedstawiać ogólne informacje na temat, którego dotyczy praca, historię i~zakres zastosowań”~\cite{bib:ZasadyPisania}. We wstępie nie powinno być rysunków, tabel i~wzorów.
%--------------------------------------
% koniec: wskazówka
%--------------------------------------

W ostatnim czasie obserwuje się intensywny rozwój [Proszę uzupełnić].

Niestety [Proszę wpisać wadę lub niedogodność istniejących rozwiązań].

W związku z~tym, ciekawym wydaje się zaprojektowanie i~zrealizowanie [Proszę uzupełnić].

\section*{Cel pracy}
\addcontentsline{toc}{section}{Cel pracy}

%--------------------------------------
% początek: wskazówka
%--------------------------------------
%„Pierwszy akapit dotyczy sformułowania problematyki i~dziedziny pracy. W~drugi akapicie opisuje się część teoretyczną – jej cele, a~w trzecim części praktycznej. Przykładowe cele: analiza istniejących metod i~technik, eksperymentalne badania różnych rozwiązań, zaprojektowanie i~wykonanie programu lub systemu komputerowego”~\cite{bib:ZasadyPisania}.
%--------------------------------------
% koniec: wskazówka
%--------------------------------------

Celem niniejszej pracy jest zaprojektowanie i~zrealizowanie [Proszę uzupełnić].

\section*{Zakres pracy}
\addcontentsline{toc}{section}{Zakres pracy}

Zakres pracy obejmuje:

\begin{itemize}
\item zebranie wiadomości z~zakresu [Proszę uzupełnić],

\item porównanie [Proszę uzupełnić],

\item opracowanie założeń projektowych dotyczących [Proszę uzupełnić],

\item implementację [Proszę uzupełnić],

\item przetestowanie [Proszę uzupełnić].
\end{itemize}

% W Rozdziale~\ref{cha:RozdzialWprowadzajacy} pracy opisano/ podsumowano/ opisano/ rozważano/ skoncentrowano się na [Proszę uzupełnić]. Natomiast w~Rozdziale [Proszę uzupełnić] opisano/ podsumowano/ opisano/ rozważano/ skoncentrowano się na [Proszę uzupełnić]. itd.