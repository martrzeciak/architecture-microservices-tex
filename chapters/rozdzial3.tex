\cleardoublepage

\chapter{gRPC-Web i komunikacja frontend-backend}
\label{cha:gRPCWeb}

\section{Ograniczenia natywnego gRPC w przeglądarkach}
\label{sec:OgraniczeniagRPC}

Przeglądarki nie oferują wystarczającej kontroli nad żądaniami HTTP/2, aby wspierać pełną specyfikację gRPC. Brak dostępu do surowych ramek HTTP/2 oraz trailing headers uniemożliwia natywną implementację gRPC w JavaScript. Ten podrozdział jest kluczowy dla zrozumienia, dlaczego potrzebne są dodatkowe rozwiązania.

\section{Architektura gRPC-Web}
\label{sec:ArchitekturagRPCWeb}

\subsection{Rola proxy w komunikacji gRPC-Web}
\label{subsec:RolaProxy}

gRPC-Web wymaga warstwy translacyjnej między przeglądarką a backendem gRPC. Ten podrozdział uzasadnia się koniecznością wyjaśnienia, jak proxy tłumaczy uproszczone żądania gRPC-Web na natywny protokół gRPC. 

\subsection{Envoy Proxy i nginx jako reverse proxy}
\label{subsec:EnvoyNginx}

Envoy i nginx to najpopularniejsze rozwiązania proxy dla gRPC-Web.

\section{Implementacja gRPC-Web w aplikacjach frontendowych}
\label{sec:ImplementacjagRPCWeb}

\subsection{Integracja z frameworkami JavaScript}
\label{subsec:IntegracjaJavaScript}

Brak natywnego wsparcia dla gRPC-Web w popularnych frameworkach frontendowych stanowi praktyczne wyzwanie. Ten podrozdział uzasadnia się potrzebą pokazania, jak integrować gRPC-Web z React, Angular czy Vue.js oraz jakie są związane z tym trudności i ograniczenia.

\subsection{Pakiety npm i konfiguracja}
\label{subsec:PakietyNPM}

Szczegółowe omówienie wymaganych pakietów npm (@types/google-protobuf, grpc-web) jest uzasadnione przez praktyczny aspekt implementacji. Ten podrozdział pomoże czytelnikom zrozumieć konkretne kroki niezbędne do wdrożenia gRPC-Web w projektach frontendowych.

\section{GraphQL Gateway dla gRPC}
\label{sec:GraphQLGateway}

Integracja GraphQL z gRPC przez grpc-graphql-gateway oferuje alternatywne podejście do ekspozycji usług gRPC.