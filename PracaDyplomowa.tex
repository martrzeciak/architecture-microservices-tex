%--------------------------------------
% autorstwo szablonu: Krystian Łapa
% aktualizacja: Krzysztof Cpałka
% zgodność z~przepisami: 2024.10.26
% wymagane narzędzia: miktex, texstudio
%--------------------------------------

\documentclass{PracaDyplomowa}

%--------------------------------------
% początek: wskazówka
%--------------------------------------
%„W pracy nie należy używać wielokrotnych spacji, ani wielokrotnych znaków nowego akapitu. Znaki interpunkcyjne takie jak przecinek (,), kropka (.), dwukropek (:), średnik (;), znak zapytania (?), wykrzyknik (!), zamknięcie dowolnego nawiasu (]})>), zamknięcie cudzysłowu (” lub ’) nie mogą być nigdy poprzedzone spacją. Bezpośrednio po wymienionych znakach może wystąpić wyłącznie spacja, znak nowego akapitu lub inny znak interpunkcyjny. Po znakach otwierających dowolnego nawiasu ([{(<) lub otwarcia cudzysłowu („ lub ‘) nigdy nie należy używać spacji. Spację używamy przed tymi znakami. Nie należy rozpoczynać akapitu od spacji – wcięcia uzyskuje się przez zastosowanie stylu „Tekst podstawowy z~wcięciem” (patrz punkt Style). Nie należy pozostawiać spacji na końcu akapitu – przed znakiem nowego akapitu. Tytułu rozdziałów i~podrozdziałów pozostawiamy bez kropki na końcu.”
% wstawianie spacji nierozdzielającej po spójnikach:~ (powoduje ona, że spójnik przechodzi do nowej linii wraz z~kolejnym wyrazem)
%--------------------------------------
% koniec: wskazówka
%--------------------------------------

%--------------------------------------
% początek: do uzupełnienia
%--------------------------------------
\author{Mariusz Trzeciak}

\title{Architektura mikroserwisów i rozwiązania chmurowe w ekosystemie .NET}

\titleeng{Microservices Architecture and Cloud Solutions in the .NET Ecosystem}

\album{133583}

\studia{stacjonarne}

\poziom{II}

\promotor{dr inż. Bartosz Kowalczyk}

\dedykacja{Niniejszym chciałbym serdecznie podziękować\\…\\za bezcenne wsparcie udzielone mi\\w trakcie trwania studiów.}
%--------------------------------------
% koniec: do uzupełnienia
%--------------------------------------

\begin{document}

\frontpage
\tableofcontents

% Rozdziały
\cleardoublepage %wymuszenie umieszczenia na nieparzystej stronie (dotyczy to tylko wstępu, pierwszego rozdziału, podsumowania i~bibliografii)
\chapter*{Wstęp}
\addcontentsline{toc}{chapter}{Wstęp}

%--------------------------------------
% początek: wskazówka
%--------------------------------------
%„Wstęp powinien przedstawiać ogólne informacje na temat, którego dotyczy praca, historię i~zakres zastosowań”~\cite{bib:ZasadyPisania}. We wstępie nie powinno być rysunków, tabel i~wzorów.
%--------------------------------------
% koniec: wskazówka
%--------------------------------------

W ostatnim czasie obserwuje się intensywny rozwój [Proszę uzupełnić].

Niestety [Proszę wpisać wadę lub niedogodność istniejących rozwiązań].

W związku z~tym, ciekawym wydaje się zaprojektowanie i~zrealizowanie [Proszę uzupełnić].

\section*{Cel pracy}
\addcontentsline{toc}{section}{Cel pracy}

%--------------------------------------
% początek: wskazówka
%--------------------------------------
%„Pierwszy akapit dotyczy sformułowania problematyki i~dziedziny pracy. W~drugi akapicie opisuje się część teoretyczną – jej cele, a~w trzecim części praktycznej. Przykładowe cele: analiza istniejących metod i~technik, eksperymentalne badania różnych rozwiązań, zaprojektowanie i~wykonanie programu lub systemu komputerowego”~\cite{bib:ZasadyPisania}.
%--------------------------------------
% koniec: wskazówka
%--------------------------------------

Celem niniejszej pracy jest zaprojektowanie i~zrealizowanie [Proszę uzupełnić].

\section*{Zakres pracy}
\addcontentsline{toc}{section}{Zakres pracy}

Zakres pracy obejmuje:

\begin{itemize}
\item zebranie wiadomości z~zakresu [Proszę uzupełnić],

\item porównanie [Proszę uzupełnić],

\item opracowanie założeń projektowych dotyczących [Proszę uzupełnić],

\item implementację [Proszę uzupełnić],

\item przetestowanie [Proszę uzupełnić].
\end{itemize}

% W Rozdziale~\ref{cha:RozdzialWprowadzajacy} pracy opisano/ podsumowano/ opisano/ rozważano/ skoncentrowano się na [Proszę uzupełnić]. Natomiast w~Rozdziale [Proszę uzupełnić] opisano/ podsumowano/ opisano/ rozważano/ skoncentrowano się na [Proszę uzupełnić]. itd.
\cleardoublepage

\chapter{Wprowadzenie do architektury mikroserwisów w ekosystemie .NET}
\label{cha:WprowadzenieDoMikroserwisow}

%--------------------------------------
% początek: wskazówka
%--------------------------------------
%Każdy rozdział powinien mieć zdanie wstępu przed przejściem do podrozdziału. Rysunki powinny być wyśrodkowane, numerowane i~powinny mieć odwołania do literatury lub oznaczenie że zostały opracowane samodzielnie.
%--------------------------------------
% koniec: wskazówka
%--------------------------------------

% W niniejszym rozdziale opisano [Proszę uzupełnić], [Proszę uzupełnić] oraz [Proszę uzupełnić].

% Sposób realizacji [Proszę uzupełnić] przedstawiono na Rys.~\ref{fig:LogoWIiSI}.

% \begin{figure}[h]
% \centering\includegraphics[width=15cm]{fig_LogoWIiSI.png}
% %\centering\includegraphics[scale=1.0]{fig_LogoWIiSI.png}
% \caption{Logo Wydziału Informatyki i~Sztucznej Inteligencji~\cite{bib:LogoWIMiI}.}
% \label{fig:LogoWIiSI}
% \end{figure}

% \noindent Można zauważyć, że Rysunek~\ref{fig:LogoWIiSI} dotyczy [Proszę uzupełnić].

% Kod służący do osadzania rysunków w~tekście pracy przedstawiono na Listingu.~\ref{lis:SposobOsadzaniaRysunkow}.

% \begin{lstlisting}[
%  caption={Kod wstawiania rysunku do tekstu pracy.},
%  label={lis:SposobOsadzaniaRysunkow}
% ]
% \begin{figure}[h]
% \centering\includegraphics[width=15cm]{NazwaPliku.png}
% \centering\includegraphics[scale=1.0]{NazwaPliku.png}
% \caption{Opis pliku.}
% \label{fig:EtykietaRysunku}
% \end{figure}
% \end{lstlisting}

% \noindent Można zauważyć, że Listing~\ref{lis:SposobOsadzaniaRysunkow} dotyczy [Proszę uzupełnić].

% Sposób realizacji [Proszę uzupełnić] można opisać następująco:

% \begin{equation}
% \label{eq:y_j}
% {{\bar y}_j}=\frac{{\sum\limits_{r=1}^N {\bar y_{j,r}^B\cdot{\mu_{{{B'}_j}}}\left({\bar y_{j,r}^B}\right)}}}{{\sum\limits_{r=1}^N {{\mu_{{{B'}_j}}}\left({\bar y_{j,r}^B}\right)}}},
% \end{equation}

% \noindent gdzie $N$ oznacza..., ${\bar y_{j,r}^B}$ oznacza..., itd. Można zauważyć, że wzór (\ref{eq:y_j}) dotyczy [Proszę uzupełnić].

% Porównanie składu tekstu przy pomocy środowisk Tex i~Word przedstawiono w~Tabeli~\ref{tab:PorownanieMozliwosciTexiWord}. Można zauważyć, że [Proszę uzupełnić].

% \begin{table}[t!] %\begin{table*}[t!] %dla dwóch kolumn
% \setlength\tabcolsep{4.5pt}
% %\def\arraystretch{0.65}
% \begin{center}
% \caption{Porównanie składu tekstu przy pomocy środowisk Tex i~Word.}
% \label{tab:PorownanieMozliwosciTexiWord}
% \footnotesize
% \begin{tabular}{|c|c|c|}
% \hline
% \textbf{Kategoria} & \textbf{Środowisko TeX} & \textbf{Środowisko Word} \\
% \hline\hline
% Typografia& \begin{tabular}{c}Wysokiej jakości układ tekstu,\\idealny do publikacji.\end{tabular} & \begin{tabular}{c}Dobre formatowanie,\\ale ograniczone\\w~precyzji typograficznej.\end{tabular} \\
% \hline
% Matematyka& \begin{tabular}{c}Doskonałe wsparcie dla równań\\i~symboli matematycznych.\end{tabular} & \begin{tabular}{c}Możliwe, ale\\trudniejsze w~edytowaniu.\end{tabular} \\
% \hline
% Praca zespołowa& \begin{tabular}{c}Wymaga synchronizacji plików;\\lepsze dla programistów.\end{tabular} & \begin{tabular}{c}Wbudowane narzędzia\\do współpracy, łatwe w~użyciu.\end{tabular} \\
% \hline
% Wizualizacja danych& \begin{tabular}{c}Ograniczone możliwości\\bez dodatkowych pakietów.\end{tabular} & \begin{tabular}{c}Silne wsparcie\\dla grafiki i~diagramów.\end{tabular} \\
% \hline
% Dokumentacja& \begin{tabular}{c}Idealne do tworzenia\\dokumentacji technicznej.\end{tabular} & \begin{tabular}{c}Dobre dla prostych dokumentów,\\łatwe do formatowania.\end{tabular} \\
% \hline
% Krzywa uczenia się& \begin{tabular}{c}Wymaga czasu\\na naukę składni.\end{tabular} & \begin{tabular}{c}Intuicyjny interfejs,\\szybka nauka\\dla początkujących.\end{tabular} \\
% \hline
% Zarządzanie referencjami& \begin{tabular}{c}Doskonałe wsparcie\\z~pakietami takimi jak BibTeX.\end{tabular} & \begin{tabular}{c}Wbudowane narzędzia,\\ale mogą być mniej elastyczne.\end{tabular} \\
% \hline
% Dostosowywanie& \begin{tabular}{c}Możliwość pełnej kontroli\\nad formatowaniem.\end{tabular} & \begin{tabular}{c}Ograniczone możliwości\\dostosowywania w~porównaniu\\do TeX.\end{tabular} \\
% \hline
% \end{tabular}
% \end{center}
% \end{table} %\end{table*} %dla dwóch kolumn

\section{Definicja i charakterystyka architektury mikroserwisowej}
\label{sec:DefinicjaMikroserwisow}

Ten podrozdział stanowił będzie fundamentalne wprowadzenie teoretyczne niezbędne dla zrozumienia całej pracy. Architektura mikroserwisowa to podejście do projektowania aplikacji jako zestawu małych, niezależnych usług, z których każda działa w swoim własnym procesie i komunikuje się przez lekkie mechanizmy. Definicja ta jest kluczowa dla czytelnika, ponieważ ustala terminologię i koncepcyjne podstawy dla dalszych rozdziałów.

\section{Porównanie z architekturą monolityczną}
\label{sec:PorownanieMikroserwisyMonolit}

Porównanie z architekturą monolityczną jest niezbędne dla ukazania kontekstu i motywacji wyboru mikroserwisów. Ten podrozdział uzasadniał się koniecznością pokazania różnic w podejściu do projektowania, wdrażania i skalowania aplikacji. Tradycyjna architektura monolityczna używa pojedynczego modelu danych dla operacji odczytu i zapisu, co może prowadzić do problemów z wydajnością i skalowalością. Mikroserwisy natomiast pozwalają na niezależne skalowanie komponentów i izolację awarii. To porównanie pomaga czytelnikowi zrozumieć, dlaczego organizacje wybierają mikroserwisy i jakie są minusy tej decyzji.

\section{Zalety i wyzwania mikroserwisów}
\label{sec:ZaletyWyzwaniaMikroserwisow}

Przedstawienie zalet i wyzwań jest kluczowe dla obiektywnej oceny architektury mikroserwisowej. Zalety obejmują zwiększoną skalowalność, elastyczność technologiczną oraz odporność na awarie. Jednak mikroserwisy wprowadzają również znaczne wyzwania, takie jak złożoność zarządzania danymi rozproszonymi, konieczność implementacji mechanizmów komunikacji międzyserwisowej oraz trudności w testowaniu. Ten podrozdział jest niezbędny dla zrównoważonego przedstawienia tematu i przygotowania czytelnika na dalsze szczegóły techniczne.

\section{Ekosystem .NET w kontekście mikroserwisów}
\label{sec:EkosystemNET}

Uzasadnienie tego podrozdziału wynika z tematu pracy skupiającego się na ekosystemie .NET. Platforma .NET Core oferuje zaawansowane narzędzia do budowy systemów mikroserwisowych, włączając integrację z nowoczesnymi technologiami chmurowymi. ASP.NET Core zapewnia natywne wsparcie dla gRPC, REST API oraz zaawansowanych wzorców architektonicznych. Ten podrozdział ustala technologiczne podstawy dla praktycznej części pracy i pokazuje, dlaczego .NET jest odpowiednim wyborem dla implementacji mikroserwisów.
\cleardoublepage

\chapter{Mechanizmy komunikacji w architekturze mikroserwisowej}
\label{cha:MechanizmyKomunikacji}

\section{Komunikacja synchroniczna}
\label{sec:KomunikacjaSynchroniczna}

\subsection{REST API w ASP.NET Core}
\label{subsec:RESTAPI}

REST pozostaje najpopularniejszym stylem architektonicznym dla komunikacji między mikroserwisami. Uzasadnienie tego podrozdziału wynika z konieczności przedstawienia tradycyjnego podejścia, które będzie później porównywane z gRPC w części badawczej pracy. REST wykorzystuje protokół HTTP/1.1 i jest intuicyjny dla programistów webowych. ASP.NET Core oferuje bogate wsparcie dla tworzenia REST API z funkcjami takimi jak automatyczna walidacja modeli, routing oraz integracja z OpenAPI.

\subsection{gRPC - architektura i implementacja}
\label{subsec:gRPC}

Ten podrozdział jest kluczowy dla aspektu badawczego pracy. gRPC wykorzystuje HTTP/2 i Protocol Buffers, co zapewnia 5-10 razy większą przepustowość niż REST. Framework ten oferuje unikalne funkcje jak np. silne typowanie interfejsów. Szczegółowe omówienie architektury gRPC jest niezbędne dla zrozumienia wyników testów wydajnościowych, które będą przeprowadzone w części empirycznej pracy.

\subsection{GraphQL jako alternatywa}
\label{subsec:GraphQL}

GraphQL zasługuje na uwagę jako alternatywne podejście do projektowania API, które może wpływać na wydajność komunikacji. Ten podrozdział uzasadnia się potrzebą przedstawienia kompletnego spektrum opcji komunikacji synchronicznej.

\section{Komunikacja asynchroniczna}
\label{sec:KomunikacjaAsynchroniczna}

\subsection{Message Brokers (RabbitMQ, Azure Service Bus)}
\label{subsec:MessageBrokers}

Komunikacja asynchroniczna jest fundamentalna dla osiągnięcia luźnego sprzężenia w architekturze mikroserwisowej. Message brokers eliminują bezpośrednie zależności między serwisami i umożliwiają implementację wzorców event-driven. RabbitMQ i Azure Service Bus są wiodącymi rozwiązaniami w ekosystemie .NET, oferującymi różne mechanizmy dostarczania wiadomości i gwarancje spójności.

\subsection{Event-driven Architecture}
\label{subsec:EventDrivenArchitecture}

Event-driven architecture stanowi kluczowy wzorzec dla mikroserwisów, umożliwiając reaktywne przetwarzanie zdarzeń. Ten podrozdział uzasadnia się koniecznością pokazania, jak zdarzenia mogą zastąpić bezpośrednie wywołania API i zwiększyć elastyczność systemu. Wzorzec ten jest szczególnie istotny w kontekście Event Sourcing, który będzie omówiony w dalszej części pracy.

\section{Komunikacja w czasie rzeczywistym}
\label{sec:KomunikacjaCzasRzeczywisty}

\subsection{WebSockets w aplikacjach internetowych}
\label{subsec:WebSockets}

WebSockets są kluczowe dla komunikacji w czasie rzeczywistym, oferując pełnodupleksową komunikację między klientem a serwerem. WebSockets eliminują ograniczenia tradycyjnego modelu request-response HTTP i umożliwiają natychmiastowe przekazywanie aktualizacji.

\subsection{Server-Sent Events (SSE)}
\label{subsec:ServerSentEvents}

SSE stanowią alternatywę dla WebSockets w scenariuszach jednostronnej komunikacji serwer-klient. Uzasadnienie tego podrozdziału wynika z potrzeby przedstawienia pełnego spektrum opcji komunikacji w czasie rzeczywistym.

\subsection{gRPC Streaming}
\label{subsec:gRPCStreaming}

gRPC oferuje unikalne możliwości streamingu, które łączą wydajność binarnej komunikacji z elastycznością streaming. Ten podrozdział uzasadnia się potrzebą pokazania, jak gRPC może konkurować z WebSockets w aplikacjach wymagających komunikacji w czasie rzeczywistym.
\cleardoublepage

\chapter{gRPC-Web i komunikacja frontend-backend}
\label{cha:gRPCWeb}

\section{Ograniczenia natywnego gRPC w przeglądarkach}
\label{sec:OgraniczeniagRPC}

Przeglądarki nie oferują wystarczającej kontroli nad żądaniami HTTP/2, aby wspierać pełną specyfikację gRPC. Brak dostępu do surowych ramek HTTP/2 oraz trailing headers uniemożliwia natywną implementację gRPC w JavaScript. Ten podrozdział jest kluczowy dla zrozumienia, dlaczego potrzebne są dodatkowe rozwiązania.

\section{Architektura gRPC-Web}
\label{sec:ArchitekturagRPCWeb}

\subsection{Rola proxy w komunikacji gRPC-Web}
\label{subsec:RolaProxy}

gRPC-Web wymaga warstwy translacyjnej między przeglądarką a backendem gRPC. Ten podrozdział uzasadnia się koniecznością wyjaśnienia, jak proxy tłumaczy uproszczone żądania gRPC-Web na natywny protokół gRPC. 

\subsection{Envoy Proxy i nginx jako reverse proxy}
\label{subsec:EnvoyNginx}

Envoy i nginx to najpopularniejsze rozwiązania proxy dla gRPC-Web.

\section{Implementacja gRPC-Web w aplikacjach frontendowych}
\label{sec:ImplementacjagRPCWeb}

\subsection{Integracja z frameworkami JavaScript}
\label{subsec:IntegracjaJavaScript}

Brak natywnego wsparcia dla gRPC-Web w popularnych frameworkach frontendowych stanowi praktyczne wyzwanie. Ten podrozdział uzasadnia się potrzebą pokazania, jak integrować gRPC-Web z React, Angular czy Vue.js oraz jakie są związane z tym trudności i ograniczenia.

\subsection{Pakiety npm i konfiguracja}
\label{subsec:PakietyNPM}

Szczegółowe omówienie wymaganych pakietów npm (@types/google-protobuf, grpc-web) jest uzasadnione przez praktyczny aspekt implementacji. Ten podrozdział pomoże czytelnikom zrozumieć konkretne kroki niezbędne do wdrożenia gRPC-Web w projektach frontendowych.

\section{GraphQL Gateway dla gRPC}
\label{sec:GraphQLGateway}

Integracja GraphQL z gRPC przez grpc-graphql-gateway oferuje alternatywne podejście do ekspozycji usług gRPC.
\cleardoublepage

\chapter{Wzorce projektowe i architektoniczne}
\label{cha:WzorceProjektowe}

\section{CQRS (Command Query Responsibility Segregation)}
\label{sec:CQRS}

CQRS jest fundamentalnym wzorcem dla mikroserwisów, pozwalającym na separację operacji odczytu i zapisu. Uzasadnienie tego podrozdziału wynika z faktu, że CQRS umożliwia niezależną optymalizację wydajności dla różnych typów operacji, co jest kluczowe w środowisku mikroserwisowym.

\section{Event Sourcing}
\label{sec:EventSourcing}

Event Sourcing jest ściśle powiązany z CQRS i oferuje unikalne korzyści jak 100\% audit logging oraz możliwość odtworzenia stanu aplikacji. Ten podrozdział uzasadnia się przez konieczność pokazania, jak Event Sourcing rozwiązuje problem atomowego aktualizowania bazy danych i wysyłania wiadomości. Wzorzec ten jest szczególnie istotny w kontekście event-driven architecture w mikroserwisach.

\section{Database-per-Service}
\label{sec:DatabasePerService}

Wzorzec Database-per-Service jest fundamentalny dla osiągnięcia niezależności mikroserwisów. Każdy serwis zarządza własną bazą danych, co zapewnia izolację, skalowalność i elastyczność technologiczną. Ten podrozdział uzasadnia się przez konieczność pokazania, jak ten wzorzec wpływa na zarządzanie danymi rozproszonymi oraz jakie wprowadza wyzwania związane z transakcjami i ich spójnością.

\section{API Gateway}
\label{sec:APIGateway}

\subsection{Porównanie Ocelot i Envoy}
\label{subsec:PorowanieOcelotEnvoy}

API Gateway jest kluczowym komponentem architektury mikroserwisowej, centralizującym routing, uwierzytelnianie i monitoring. Porównanie Ocelot (natywnego dla .NET) z Envoy uzasadnia się przez praktyczne potrzeby wyboru odpowiedniego rozwiązania. Envoy oferuje lepszą wydajność (100k RPS vs 10k RPS dla Ocelot) ale wymaga dodatkowej konfiguracji.

\subsection{Circuit Breaker Pattern}
\label{subsec:CircuitBreaker}

Circuit Breaker jest kluczowy dla zapewnienia odporności systemu mikroserwisowego na awarie. Ten podrozdział uzasadnia się przez konieczność pokazania, jak wzorzec ten zapobiega kaskadowym awariom.

\cleardoublepage

\chapter{Orkiestracja kontenerów i rozwiązania chmurowe}
\label{cha:OrkiestracjaKontenerow}

\section{Kubernetes w ekosystemie .NET}
\label{sec:Kubernetes}

\subsection{Azure Kubernetes Service (AKS)}
\label{subsec:AzureAKS}

AKS jest wiodącą platformą dla deploynetu mikroserwisów .NET w chmurze Azure. Ten podrozdział uzasadnia się przez praktyczne potrzeby wdrażania i zarządzania mikroserwisami w środowisku produkcyjnym. AKS oferuje zarządzaną płaszczyznę kontrolną Kubernetes oraz integrację z innymi usługami Azure.

\subsection{Wdrażanie z Helm Charts}
\label{subsec:HelmCharts}

Helm Charts stanowią standard dla wdrażania aplikacji w Kubernetes. Ten podrozdział uzasadnia się przez konieczność pokazania, jak automatyzować deployment mikroserwisów oraz zarządzać ich konfiguracją w różnych środowiskach. Helm umożliwia wersjonowanie i przywracanie poprzednich wersji wdrożeń, co jest kluczowe dla CI/CD.

\section{Service Mesh}
\label{sec:ServiceMesh}

\subsection{Linkerd i Istio}
\label{subsec:LinkerdIstio}

Service Mesh rozwiązuje problemy komunikacji, bezpieczeństwa i observability w architekturze mikroserwisowej. Porównanie Linkerd i Istio uzasadnia się przez znaczące różnice w wydajności - Linkerd dodaje 40-400\% mniej latencji niż Istio. Ten podrozdział jest istotny dla zrozumienia, jak Service Mesh wpływa na wydajność komunikacji między mikroserwisami.

\subsection{Distributed Tracing}
\label{subsec:DistributedTracing}

Distributed Tracing jest niezbędny dla debugowania i monitorowania systemów mikroserwisowych.

\section{CI/CD Pipeline dla mikroserwisów}
\label{sec:CICD}

CI/CD dla mikroserwisów wprowadza unikalne wyzwania związane z niezależnym deploynmentem serwisów. Azure DevOps oferuje dedykowane narzędzia dla tego celu.

\cleardoublepage

\chapter{Metodologia badań i projekt eksperymentu}
\label{cha:MetodologiaBadan}

Uzasadnienie części empirycznej (Rozdziały 6-8).
Te rozdziały realizują główny cel badawczy pracy poprzez praktyczne testy wydajnościowe w różnych scenariuszach obciążenia i rozmiarów danych.

\section{Cele i hipotezy badawcze}
\label{sec:CeleHipotezy}

\section{Projekt aplikacji testowej}
\label{sec:ProjektAplikacji}

\subsection{Architektura systemu testowego}
\label{subsec:ArchitekturaTestowa}

\subsection{Implementacja mikroserwisów w .NET}
\label{subsec:ImplementacjaMikroserwisow}

\section{Scenariusze testowe}
\label{sec:ScenariuszeTestowe}

\subsection{Testy obciążeniowe}
\label{subsec:TestyObciazeniowe}

\subsection{Testy wydajnościowe}
\label{subsec:TestyWydajnosciowe}

\section{Narzędzia i środowisko testowe}
\label{sec:NarzedziaTestowe}

\cleardoublepage

\chapter{Porównawcza analiza wydajności gRPC vs REST}
\label{cha:AnalizaWydajnosci}

\section{Przeprowadzenie testów wydajnościowych}
\label{sec:TestyWydajnosciowe}

\subsection{Różne rozmiary payload'ów}
\label{subsec:RozneRozmiary}

\subsection{Różne poziomy obciążenia}
\label{subsec:RozneObciazenia}

\subsection{Modele komunikacji (synchroniczna, strumieniowa)}
\label{subsec:ModelKomunikacji}

\section{Analiza wyników}
\label{sec:AnalizaWynikow}

\subsection{Przepustowość (transakcje na sekundę)}
\label{subsec:Przepustowosc}

\subsection{Opóźnienia i czas odpowiedzi}
\label{subsec:Opoznienia}

\subsection{Wykorzystanie zasobów systemowych}
\label{subsec:WykorzystanieZasobow}

\section{Porównanie z WebSockets i gRPC-Web}
\label{sec:PorowanieWebSockets}

\cleardoublepage

\chapter{Studium przypadku: aplikacja eShop}
\label{cha:StudiumPrzypadku}

\section{Architektura aplikacji referencyjnej}
\label{sec:ArchitekturaReferencyjna}

\section{Implementacja wybranych wzorców}
\label{sec:ImplementacjaWzorcow}

\section{Testowanie i walidacja rozwiązań}
\label{sec:TestowanieWalidacja}

\cleardoublepage
\chapter{Podsumowanie}

%--------------------------------------
% początek: wskazówka
%--------------------------------------
%„Dyskusja nad dalszym rozwojem pracy. Wnioski. Omówienie wyników. Co zrobiono w~pracy i~jakie uzyskano wyniki? Czy i~w jakim zakresie praca stanowi nowe ujęcie problemu? Sposób wykorzystania pracy (publikacja, udostępnienie instytucjom, materiał źródłowy dla studentów). Co uważa autor za własne osiągnięcia?”~\cite{bib:ZasadyPisania}. Dodatkowo warto zaznaczyć czy udało się osiągnąć założony cel pracy.
%--------------------------------------
% koniec: wskazówka
%--------------------------------------

W ramach niniejszej pracy:

\begin{itemize}
\item zebrano wiadomości z~zakresu [Proszę wpisać tematykę],
\item opisano istniejące rozwiązania z~zakresu [Proszę wpisać],
\item zaprojektowano autorski system informatyczny [Proszę wpisać pełną nazwę systemu],
\item zaimplementowano autorski system informatyczny [Proszę wpisać pełną nazwę systemu],
\item przetestowano [Proszę wpisać, co przetestowano w~pracy].
\end{itemize}

Zaprojektowany i~zrealizowany w~pracy autorski [Proszę wpisać pełną nazwę systemu] charakteryzuje się tym, że:

\begin{itemize}
\item ... [Proszę wpisać zaletę systemu],
\item ... [Proszę wpisać zaletę systemu],
\item .. [Proszę wpisać zaletę systemu].
\end{itemize}

Zaprojektowany i~zrealizowany w~pracy autorski [Proszę wpisać pełną nazwę systemu] można wykorzystać m.in. do:

\begin{itemize}
\item ... [Proszę wpisać potencjalną możliwość zastosowania],
\item ... [Proszę wpisać potencjalną możliwość zastosowania],
\item ... [Proszę wpisać potencjalną możliwość zastosowania].
\end{itemize}

Zaprojektowany i~zrealizowany w~pracy autorski [Proszę wpisać pełną nazwę systemu] można w~przyszłości rozbudować o~następujące funkcje:

\begin{itemize}
\item ... [Proszę wpisać potencjalną możliwość rozbudowy],
\item ... [Proszę wpisać potencjalną możliwość rozbudowy],
\item ... [Proszę wpisać potencjalną możliwość rozbudowy].
\end{itemize}

Podsumowując można stwierdzić, że [Proszę wpisać pełną nazwę systemu], ze względu na [Proszę raz jeszcze wymienić podstawową zaletę rozwiązania], może być niezwykle użyteczny w~[Proszę krótko wpisać obszar zastosowań].


\cleardoublepage
\begin{thebibliography}{99}
\addcontentsline{toc}{chapter}{\bibname}

%--------------------------------------
% początek: wskazówka
%--------------------------------------
%formatowanie literatury:
%Książka, artykuł:
%Nazwisko Pierwsza_litera_imienia., Tytuł_italikiem, źródło_informacji (wydawnictwo), rok wydania, strony
%strona WWW:
%Autorzy (jeśli podani), Tytuł, dokładny adres, stan na dzień: data
%Przykłady: 
%Amborski K., Teoria sterowania, Warszawa: PWN, 1987. str. 80-100
%Cisco Systems, Dynamic ISL (DISL), http://www.cisco.com/en/US/tech/tk389/tk390/tk162/tech_protocol_home.html, stan na dzień: 20.12.2004
%--------------------------------------
% koniec: wskazówka
%--------------------------------------

\bibitem{bib:ZasadyPisania} Zasady pisania prac dyplomowych, https://wiisi.pcz.pl/student-wiisi/vademecum-studenta/praca-dyplomowa, stan na dzień: 26.10.2024

\bibitem{bib:LogoWIMiI} Strona internetowa Wydziału Informatyki i~Sztucznej Inteligencji, https://wiisi.pcz.pl, stan na dzień: 26.10.2024

\end{thebibliography}

\chapter*{Dodatek. Zawartość dołączonej płyty}
\addcontentsline{toc}{chapter}{Dodatek. Zawartość dołączonej płyty}

Do niniejszej pracy dołączono płytę z~następującą zawartością:

\begin{itemize}
\item Dokument pracy w~formatach tex i~pdf.

\item Kod źródłowy zaprojektowanego i~zrealizowanego w~ramach pracy systemu.
\end{itemize}

\listoffigures
\addcontentsline{toc}{chapter}{\listfigurename}

\listoftables
\addcontentsline{toc}{chapter}{\listtablename}

\lstlistoflistings
\addcontentsline{toc}{chapter}{\lstlistingname}

\chapter*{Streszczenie}
\addcontentsline{toc}{chapter}{Streszczenie}

Praca pt. \textit{„Tytuł pracy”} nawiązuje do bardzo aktualnych zagadnień współczesnej informatyki, a~mianowicie do [Proszę wpisać ogólną tematykę pracy]. W~ostatnim czasie obserwuje się intensywny rozwój [Proszę wpisać ogólne wprowadzenie do tematu]. Jednocześnie [Proszę wpisać wady lub niedogodności takiego rozwiązania]. W~rozważanej pracy podjęto zatem próbę rozwiązania tego problemu. W~szczególności zebrano wiadomości z~zakresu [Proszę wpisać tematykę], porównano [Proszę wpisać, jakie rozwiązania działające porównano], opracowano [Proszę wpisać, co opracowano w~pracy], zrealizowano [Proszę wpisać, co zrealizowano w~pracy], przetestowano [Proszę wpisać, co przetestowano w~pracy]. Zaprojektowany i~zaimplementowany system charakteryzuje się następującymi cechami: [Proszę uzupełnić], [Proszę uzupełnić], [Proszę uzupełnić]. Dzięki temu zaprojektowany i~zaimplementowany system może być wykorzystany w~praktyce np. do: [Proszę wpisać potencjalną możliwość zastosowania], [Proszę wpisać potencjalną możliwość zastosowania], [Proszę wpisać potencjalną możliwość zastosowania].

Do realizacji praktycznej części pracy wykorzystano najnowsze narzędzia i~technologie informatyczne, w~tym m.in. [Proszę wpisać nazwę technologii], [Proszę wpisać nazwę technologii], [Proszę wpisać nazwę technologii].

Zaprojektowany i~zaimplementowany system można w~przyszłości przykładowo rozbudować o~następujące funkcje: [Proszę wpisać], [Proszę wpisać] i~[Proszę wpisać].

\chapter*{Summary}
\addcontentsline{toc}{chapter}{Summary}

Proszę umieścić tłumaczenie polskiej wersji streszczenia.

\chapter*{Słowa kluczowe}
\addcontentsline{toc}{chapter}{Słowa kluczowe}

\noindent informatyka;

\noindent zaprojektowanie autorskiego systemu informatycznego;

\noindent implementacja autorskiego systemu informatycznego;

\noindent przetestowanie autorskiego systemu informatycznego;

\noindent [Proszę wpisać słowo dotyczące tematyki pracy];

\noindent [Proszę wpisać słowo dotyczące kluczowej funkcjonalności systemu];

\noindent [Proszę wpisać słowo dotyczące użytej technologii];

\noindent [Proszę wpisać słowo dotyczące użytej technologii];

\noindent [Proszę wpisać słowo dotyczące użytej technologii]

\oswiadczenie % tego elementu ma nie być w~spisie treści

\end{document}